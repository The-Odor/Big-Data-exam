% Do not modify these
\documentclass[fleqn,10pt]{wlscirep}
\usepackage[utf8]{inputenc}
\usepackage[T1]{fontenc}



% -- Insert any custom LaTeX packages here --

% \package{natbib} % <-- Required for the Chicago citation style
% \package{apacite} % <-- Required for the APA citation style
% If you decide to use one of the styles above, remember to change the \bibliographystyle{} at the bottom of the document too!

\usepackage{listings} % <-- Required if you want to display program source code in your paper.

\usepackage{hyperref}
\hypersetup{
    colorlinks=true,
    linkcolor=blue,
    filecolor=magenta,      
    urlcolor=cyan,
}
 
\urlstyle{same}

% -- End of custom LaTeX packages --


% Fill in your title
\title{Exploring Bitcoin }

% Do not modify the author tag below, just let it be blank
\author{}

% Fill in assignment abstract
\begin{abstract}
In this assignment we parsed a dataset, in the form of XMLs, through a Hadoop- and Pig apache. We learned how to write MapReduce jobs and Pig scripts, and how effective it is on our dataset based on XML files from Bitcoin stack exchange. 

\end{abstract}


% Do not modify the following two lines
\begin{document}
\include{cover}


% Insert data for the hand-in's cover page
\makecoverpage{
	master_of 		 = \par{Applied Computer Science},  % Use either: Applied Computer Science | Human-Computer Interaction
	assignment_title = \par{ Exploring Bitcoin} ,  % Title of your assignment
	course_code    	 = \par{MA120},  % Course code (ex. MA110)
	course_name      = \par{Big Data},  % Course name (ex. Systems Development)
	due_date		 = \par{18 Oktober},  % Due date
	student_name     = \par{Theodore Midtbø Alstad ; Howie Chen},  % Your name (or names, if group – separate names with ; semicolon)
	student_number   = \par{865317 ; 866354},  % Your student ID number (or numbers, if group – separate ID numbers with ; semicolon)
	group_size		 = 2, % Number of group members (used for the declaration text)
}


% Do not modify the following two lines
\flushbottom
\maketitle


% --INTRODUCTION--
\section*{Introduction}
We choose to work together because both of us have python background, therefor we choose python as main programming language. We explored Bitcoin as given dataset from \url{archive.org/download/stackexchange/bitcoin.stackexchange.com.7z}  through Apaches: Hadoop and Pig. 

% --FUNCTIOS %-- 
\section*{Main functions}

% --TASK1--
\section*{Task 1 Warmup}
%This part of the task is about get to know how hadoop- and pig apache works and how to parse XML  files throught python code. 
\subsection*{1a) WordCount}
\textbf{Assumption}: Count the words in body of questions PostTypeID="1"  in the \textit{Posts.xml}  file. The result should be how many times a word occur in the body of questions.\\ \\
\textbf{Implementation}Hello hello \\ \\
\textbf{Notes/Reflection} bye bye  \\ \\

\subsection*{1b) Unique words}
\textbf{Assumption}: Write a MapReduce job where the result should be unique words in the titles PostTypeID="1" in the \textit{Posts.xml} File. \\ \\
\textbf{Implementation}  \\ \\
\textbf{Notes/Reflection}

\subsection*{1c) MoreThan10}
\textbf{Assumption}: Write a simple python code to check how many words there is in the title in the \textit{Posts.xml}. The result should output how many titles are longer than 10 words.  \\ \\
\textbf{Implementation}  \\ \\
\textbf{Notes/Reflection}

\subsection*{1d) Stopwords}
\textbf{Assumption}: Write a simple python code based on task 1a to exclude \href{https://raw.githubusercontent.com/naimdjon/stopwords/master/stopwords.txt}{stopwords} from body of questions PostTypeID="1" in the \textit{Posts.xml}. The output should be text file without any stopwords. \\ \\
\textbf{Implementation}  \\ \\
\textbf{Notes/Reflection}

\subsection*{1e) Pig top 10}
\textbf{Assumption}: Write a pig script to select top 10 listed words after removing the stopwords from \textit{Posts.xml}. The output should print out top 10 listed words.\\ \\
\textbf{Implementation}  \\ \\
\textbf{Notes/Reflection}

\subsection*{1f) Tags}
\textbf{Assumption}: Write a MapReduce job to create a dictionary over unique tags in  \textit{Posts.xml}. \\ \\
\textbf{Implementation}  \\ \\
\textbf{Notes/Reflection}

% --TASK2--
\section*{Task 2 Discover}
%This part of the task is about to looking through several 

\subsection*{2a Counting) }
\textbf{Assumption}: Write a simple python code to list how many unique users there are in \textit{Users.xml}. The output should \\ \\
\textbf{Implementation}  \\ \\
\textbf{Notes/Reflection}

\subsection*{2b Unique users) }
\textbf{Assumption}:  \\ \\
\textbf{Implementation}  \\ \\
\textbf{Notes/Reflection}

\subsection*{2c Top miners) }
\textbf{Assumption}:  \\ \\
\textbf{Implementation}  \\ \\
\textbf{Notes/Reflection}

\subsection*{2d Top questions) }
\textbf{Assumption}:  \\ \\
\textbf{Implementation}  \\ \\
\textbf{Notes/Reflection}

\subsection*{2e Favorite questions) }
\textbf{Assumption}:  \\ \\
\textbf{Implementation}  \\ \\
\textbf{Notes/Reflection}

\subsection*{2f Average answers) }
\textbf{Assumption}:  \\ \\
\textbf{Implementation}  \\ \\
\textbf{Notes/Reflection}

\subsection*{2g Countries) }
\textbf{Assumption}:  \\ \\
\textbf{Implementation}  \\ \\
\textbf{Notes/Reflection}

\subsection*{2h Names) }
\textbf{Assumption}:  \\ \\
\textbf{Implementation}  \\ \\
\textbf{Notes/Reflection}

\subsection*{2i) Answers }
\textbf{Assumption}:  \\ \\
\textbf{Implementation}  \\ \\
\textbf{Notes/Reflection}

% --TASK3--
\section*{Task 3 Numbers}

\subsection*{3a) Bigram }
\textbf{Assumption}:  \\ \\
\textbf{Implementation}  \\ \\
\textbf{Notes/Reflection}

\subsection*{3b) Trigram }
\textbf{Assumption}:  \\ \\
\textbf{Implementation}  \\ \\
\textbf{Notes/Reflection}

\subsection*{3c) Combiner }
\textbf{Assumption}:  \\ \\
\textbf{Implementation}  \\ \\
\textbf{Notes/Reflection}

\subsection*{3d) Useless}
\textbf{Assumption}:  \\ \\
\textbf{Implementation}  \\ \\
\textbf{Notes/Reflection}

% --TASK4--
\section*{Task 4 Search engine}
\subsection*{4a) Title index}
\textbf{Assumption}:  \\ \\
\textbf{Implementation}  \\ \\
\textbf{Notes/Reflection}

\section*{Conclusion}
% Do not modify this last lines
\end{document}